\section{Ejercicios Avanzados}

\subsection{Ejercicio 2.1}

\begin{multicols}{2}

\noindent\textbf{MER 1}

\noindent Persona(\PK{dni}, nombre, direccion, telefono, obraSocial, nroAfiliado, nombreSindicato, fechaIngreso, puesto)

\noindent tieneHijos(\FPK{dni1}, \FPK{dni2})

\vspace*{0.5cm}
\noindent\textbf{Restricción:} Una persona no puede ser su propio hijo ($dni1 \neq dni2$)
\vspace*{1cm}

\noindent\textbf{MER2}

\noindent Persona(\PK{dniPersona}, nombre, direccion, telefono, obraSocial, nroAfiliado, nombreSindicato, fechaIngreso, puesto)

\noindent Hijos(\PK{dniHijo}, nombre, direccion)

\noindent tiene(\FPK{dniPersona}, \FPK{dniHijo})

\columnbreak
\noindent\textbf{MER3}

\noindent Persona(\PK{dni}, nombre, direccion, tipo)

\noindent Empleado(\FPK{dni}, nombreSindicato, nroAfiliado, obraSocial, telefono, fechaIngreso, puesto)

\noindent Hijos(\FPK{dni})

\noindent tiene(\FPK{dni1}, \FPK{dni2})

\vspace*{0.5cm}
\noindent\textbf{Restriccion:} \textit{dn1} debe ser el dni de un empleado y \textit{dn2} el dni de un hijo. (dn1 debe tener una entrada en la tabla empleado y dn2 una en la tabla hijo)
\end{multicols}

\newpage
\begin{landscape}
\subsection{\red{Ejercicio 2.2 - Revisar}}
\begin{center}
	\scalebox{0.75}{
		\begin{tikzpicture}[.style={DER}]	
		
		%% Persona
		\node[entity]				(persona)													{Persona};
		\node[attributePK]			(personaDni)			[above=of persona]					{\PK{dni}};
		\node[attribute]			(personaApellido)		[above left=of persona]				{apellido};
		\node[attribute]			(personaNombre)			[left=of persona]					{nombre};
		\node[attribute]			(personaDireccion)		[below left=of persona]				{direccion};
		\node[attribute]			(personaTelefono)		[below=of persona]					{telefono};
		
		%% InheritaEntidadesnce Persona
		\node[inheritance]			(inheritancePersona)	[right=of persona]					{D};
		\node[entity]				(cliente)				[below right=of inheritancePersona]	{Cliente};
		
		%% Vendedor
		\node[entity]				(vendedor)				[above right=of inheritancePersona]	{Vendedor};			
		
		%% Realizó y Partició en
		\node[weak relationship]	(realizo)				[right=of vendedor]					{realizó};
		\node[relationship]			(participoEn)			[right=of cliente]					{participó en};
		
		%% Venta
		\node[weak entity]			(venta)					[right=of participoEn]				{Venta};
		\node[attributePK]			(ventaIdVenta)			[above right=of venta]				{\PK{idVenta}};
		\node[attribute]			(ventaFecha)			[right=of venta]					{fecha};
		
		%% pertenece A
		\node[relationship]			(tiene)					[below=of cliente]					{tiene};
		\node[attribute]			(tieneHasta)			[left=of tiene] 					{hasta};
		\node[attribute]			(tieneDesde)			[above =of tieneHasta] 				{desde};
		
		%% Vehiculos e inheritance
		\node[entity]				(usado)					[below=of tiene]					{Usado};
		\node[inheritance]			(inheritanceVehiculo)	[below left=of usado]				{D};
		\node[entity]				(nuevo)					[below right=of inheritanceVehiculo]{Nuevo};
		
		\node[entity]				(vehiculo)				[left=of inheritanceVehiculo]		{Vehiculo};
		\node[attributePK]			(vehiculoMatricula)		[above left=of vehiculo]			{\PK{matricula}};
		
		\node[relationship]			(cedidoEn)				[right=of usado]					{cedido en};
		\node[relationship]			(vendidoEn)				[right=of nuevo]					{vendido en};
		
		\node[attribute]			(usadoTasacion)			[left=of usado]						{tasación};	
		\node[attribute]			(nuevoPrecio)			[above=of nuevo]					{precio};
		
		%%Modelo
		\node[relationship] 		(esModelo)				[below=of vehiculo]					{es};
		\node[entity] 				(modelo)				[below=of esModelo] 				{Modelo};
		
		\node[attributePK] 			(modeloNombre) 			[above left=of modelo]				{\PK{nombre}};
		\node[attribute]			(modeloCilindrada) 		[left=of modelo] 					{cilindrada};
		\node[attributePK] 			(modeloMarca)			[below left=of modelo] 				{\PK{marca}};
		
		%%Opciones
		\node[relationship]			(puedeTener)			[below=of modelo]					{puede tener};
		\node[attribute]			(puedeTenerPrecio)		[below=of puedeTener]				{precio};
		\node[entity]				(opcion)				[right=of puedeTener]				{Opcion};
		\node[attributePK]			(opcionNombre)			[below=of opcion]					{\PK{nombreOp}};
		\node[attribute]			(opcionDescripcion)		[below right=of opcion]				{descripción};
		
		\node[relationship]			(conOpciones)			[below=of nuevo]					{con};
		
		%%Ejes
		\path
		%%Atributos
		(persona)				edge	(personaDni)
		(persona)				edge	(personaApellido)
		(persona)				edge	(personaNombre)
		(persona)				edge	(personaDireccion)
		(persona)				edge	(personaTelefono)
		
		(inheritancePersona)	edge 	(persona)
		(inheritancePersona)	edge 	(vendedor)
		(inheritancePersona)	edge 	(cliente)
		
		(venta)					edge	(ventaIdVenta)
		(venta)					edge	(ventaFecha)
		
		(vehiculo)				edge 	(vehiculoMatricula)
		
		
		(inheritanceVehiculo)	edge	(vehiculo)
		(inheritanceVehiculo)	edge	(nuevo)
		(inheritanceVehiculo)	edge	(usado)
		
		(nuevo)					edge	(nuevoPrecio)
		(usado)					edge	(usadoTasacion)			
		
		
		(tiene)					edge	(tieneDesde)
		(tiene)					edge	(tieneHasta)
		
		(modelo)				edge	(modeloNombre)
		(modelo)				edge	(modeloCilindrada)
		(modelo)				edge	(modeloMarca)
		
		(opcion)				edge	(opcionNombre)
		(opcion)				edge	(opcionDescripcion)
		
		(puedeTener)			edge	(puedeTenerPrecio)
		
		%%Relaciones
		(participoEn)		edge						node[above, near end]		{N}	(cliente)
		(participoEn)		edge[out=east, in=160]						node[above, near end]		{M}	(venta)
		
		(realizo.east)		edge[in=north, out=east]	node[right, pos=0.95]		{N}	(venta.north)
		
		(tiene)				edge						node[right, near end]		{M}	(usado)	
		
		(cedidoEn)			edge						node[above, near end]		{1}	(usado)
		
		(vendidoEn)			edge						node[above, near end]		{1}	(nuevo)
		
		(vendidoEn)			edge[in=220, out=east]		node[below right, pos=0.95]	{1}	(venta)
		
		(esModelo)			edge 						node[right, near end]		{N}	(vehiculo)
		(esModelo)			edge 						node[right, near end]		{1}	(modelo)
		
		(puedeTener)		edge						node[right, near end]		{N}	(modelo)
		(puedeTener)		edge						node[above, near end]		{M}	(opcion)
		;
		
		\path[rightOptional]
		(realizo)			edge						node[above, near end]		{1}	(vendedor)
		
		(tiene)				edge						node[right, near end]		{N}	(cliente)
		
		(cedidoEn.east)		edge[out=east, in=west]		node[below, pos=0.95]	{1}	(venta)
		
		(conOpciones)		edge						node[left, near end]		{M}	(nuevo)
		(conOpciones)		edge[in=north, out=south]	node[above left, near end]		{N}	(opcion)
		;
		\end{tikzpicture}
	}
\end{center}

\begin{multicols}{2}
	\textbf{MER}
	
	\noindent Persona(\PK{dni}, nombre, apellido, direccion, telefono, tipo)
	
	\noindent Venta(\PK{idVenta}, fecha, \FPK{dni}, \FK{matricula})
	
	\noindent  Vehiculo(\PK{matricula}, tipo, \FPK{nombre, marca})
	
	\noindent Usado(\FPK{matricula}, tasacion, \FK{idVenta})
	
	\noindent Nuevo(\FPK{matricula}, precio)
	
	\noindent Modelo(\PK{nombre, marca}, cilindrada)
	
	\noindent Opcion(\PK{nombreOp}, descripcion)
	
	\noindent participoEn(\FPK{dni}, \FPK{idVenta})
	
	\noindent tiene(\FPK{dni}, \FPK{matricula}, desde, hasta)

	\noindent puedeTener(\FPK{nombre, marca}, \FPK{nombreOp}, precio)
	
	\noindent con(\FPK{matricula}, \FPK{nombreOp})

	\columnbreak
	\textbf{Restricciones}
	\begin{itemize}
		\item \textit{cedido.matricula} debe ser un auto usado y debe pertenecer al cliente que participó en la venta \textit{cedido.idVenta}. 
		\item \textit{tiene.hasta} debe ser menor o igual a la fecha en la que ocurrió la venta en la que el cliente cedió el auto.
		\item \textit{vendidoEn.matricula} debe ser la matricula de un auto nuevo.
		\item \textit{nuevo.precio} debe tener en cuenta el precio de cada opcion agregada al auto.
	\end{itemize}	
	
	\textbf{\red{Nota}}
	\begin{itemize}
		\item Acá no puedo usar una ternaria Usuario - Usado - Venta porque hay ventas en las que el usuario no cede ningún auto. Si pusiesemos una ternaria, entonces tendriamos una tabla en la que un atributo quedaría nulo la mayoría de las veces.
	\end{itemize}
\end{multicols}

\end{landscape}


\newpage
\subsection{Ejercicio 2.3}
\textbf{MER}

Producto(\PK{idProducto}, descripcion)

Distribuidor(\PK{numDistribuidor}, nombre)

Area(\PK{idArea}, nombre)

Local(\PK{numLocal}, direccion)

Deposito(\PK{numDeposito}, \FPK{Local} capacidad)

Empleado(\PK{CUIL}, fechaIngreso, tipoContratacion)

EmpleadoEfectivo( \FPK{CUIL}, cantHijos)

EmpleadoEfectConGremio(\FPK{CUIL}, fechaAfiliacionGremio)

EmpleadoEfectConPrepaga(\FPK{CUIL}, numAfiliado)

trabajaEn(\FPK{numLocal}, \FPK{CUIL}, \FK{idArea}, cantHoras)

distribuidoPor(\FPK{idProducto}, \FPK{numDistribuidor}, \FK{idArea})

\newpage
\subsection{\red{Ejercicio 2.4 - Revisar}}

\vspace*{-1.25cm}
\begin{center}
	\scalebox{0.8}{
		\begin{tikzpicture}[.style={DER}]	
		
%% Entidades y relaciones
\node[entity]		(departamento)										{Departamento};
\node[relationship]	(trabajaEn)			[above=of departamento]			{trabaja en};

\node[]		(aTrabajaEn)		[above=of trabajaEn]			{};
\node[entity]		(empleado)			[above=of aTrabajaEn]						{Empleado};
\node[inheritance]	(inhEmpleado)		[right=of empleado]							{D};
\node[entity]		(administrativo)	[above right=of inhEmpleado, yshift=2cm]	{Administrativo};
		
\node[entity]		(docente)			[below right=of inhEmpleado]	{Docente};
\node[]				(rDocente)			[right=of docente]				{}; %Auxiliar
\node[inheritance]	(inhDocente)		[right=of rDocente]				{D};
\node[entity]		(aTiempoCompleto)	[above right=of inhDocente]		{A tiempo completo};
\node[entity]		(aTiempoParcial)	[below right=of inhDocente]		{A tiempo parcial};
		
\node[inheritance]	(inhAdministrativo)	[right=of administrativo]		{D};
\node[entity]		(coordinador)		[right=of inhAdministrativo]	{Coordinador};
\node[entity]		(secretario)		[below=of coordinador]			{Secretario};
\node[entity]		(tecnico)			[above=of coordinador]			{Técnico};
		
\node[]				(ltrabajaEn)		[left=of trabajaEn, xshift=-2cm, yshift=1.75cm]				{}; %Auxiliar
\node[relationship]	(perteneceA)		[left=of ltrabajaEn]			{pertenece a};
\node[]				(lperteneceA)		[left=of perteneceA]			{}; %Auxiliar
\node[entity]		(estudiante)		[above=of perteneceA]			{Estudiante};
\node[inheritance]	(inhEstudiante)		[above=of estudiante]			{O};
\node[entity]		(regular)			[above=of inhEstudiante]	{Regular};
\node[entity]		(graduado)			[right=of inhEstudiante]	{Graduado};

\node[relationship]	(encargadoDe)		[below=of docente]				{encargado de};
\node[relationship]	(egresadoDe)		[right =of estudiante]			{egresado de};
			
\node[weak relationship]	(realiza)			[left=of perteneceA, yshift=-0.5cm]			{realiza};
\node[weak entity]		(inscripcion)		[below=of realiza]				{Inscripcion};
\node[relationship]		(para)				[below=of inscripcion]	
		{para};

\node[entity]			(curso)				[below=of para]			{Curso};

\node[relationship]		(dictadoPor)					[below=of curso]				{dictado por};

\node[entity]			(semestre)			[left=of dictadoPor]		{Semestre};
	
\node[]					(pDictadoPorADocente)	[right=of dictadoPor, xshift=15cm]			{};

\node[relationship]		(seDivideEn)			[below=of departamento]				{se dividen en};			
	\node[entity]			(seccion)			[below left=of seDivideEn, yshift=0.5cm]				{Seccion};			
		
\node[relationship]		(dictadoEn)				[left=of seccion]	{dictado en};
%% Atributos Coordinador
\node[attribute]	(coordinadorEmail)	[right=of coordinador]	{email};

%% Atributos curso
\node[attributePK]	(cursoIdCurso)					[above left=of curso, yshift=-1cm]	{\PK{codCurso}};
\node[attribute]	(cursoNombre)					[above right=of curso]		{nombre};
\node[attribute]	(cursoCreditos)					[below=of cursoNombre]		{creditos};

%% Atributos Departamento
\node[attributePK]	(departamentoIdDpto)	[right=of departamento, yshift=-1cm]	{\PK{idDpto}};
\node[attribute]	(departamentoNombre)	[below=of departamentoIdDpto]			{nombre};

%		%% Atributos Docente
\node[attribute]	(docenteEmail)		[above=of docente]				{email};
\node[attribute]	(docentePaginaWeb)	[above right=of docente] 		{paginaWeb};

%		%% Atributos Empleado
\node[attributeM]	(empleadoTelefono)	[above right=of empleado, xshift=-0.5cm]		{telefonos};
\node[attribute]	(empleadoNombre)	[above=of empleado, yshift=0.5cm]		{nombre};
\node[attributePK]	(empleadoCodigo)	[above left=of empleado,xshift=0.8cm, yshift=1cm]		{\PK{codigoEmp}};

%% Atributos Estudiante
\node[attributePK]	(estudianteCodigoEstudiante)	[above left=of estudiante] 		{\PK{codEstudiante}};
\node[attribute]	(estudianteNombre)				[below=of estudianteCodigoEstudiante, yshift=0.5cm] 			{nombre};
\node[attribute]	(estudianteMail)				[below=of estudianteNombre,yshift=0.5cm]		{mail};

%% Atributos Inscripcion
\node[attributePK]	(inscripcionNroInscripcion)		[above left=of inscripcion]	{\PK{nroInscripcion}};
\node[attribute]		(paraNota)				[above left=of para, yshift=-0.25cm]		{nota};

%% Atributos semestre
\node[attributePK]	(semestreIdSemestre)			[below left=of semestre]				{\PK{idSemestre}};
\node[attribute]	(semestreFechaInicio)			[below =of semestre]		{fechaInicio};
\node[attribute]	(semestreFechaFin)				[below right=of semestre]				{fechaFin};


%% Atributos Tecnico
\node[attribute]	(tecnicoNivelEstudio)			[right=of tecnico]				{nivelEstudio};
%				
%	
%% Atributos egresadoDe
\node[attribute]	(egresadoDeAnioEgreso)			[below left=of egresadoDe,xshift=.9cm, yshift=0.5cm]		{añoEgreso};

%%Atributos relación perteneceA
\node[attribute]	(perteneceAAnioIngreso)			[below right=of perteneceA, xshift=-0.5cm, yshift=0.5cm]			{añoIngreso};

			
%% Atributos sección
\node[attributePK]	(seccionIdSeccion)				[above=of seccion]		{\PK{idSeccion}};
\node[attribute]	(seccionNombre)					[above left=of seccion]				{nombre};
%		

%% AUXILIAR
\node[]				(lcodCurso)		[left=of cursoIdCurso]	{};

\path
	%%Atributos
	(coordinador)		edge	(coordinadorEmail)

	(curso)				edge	(cursoIdCurso)
	(curso)				edge	(cursoNombre)
	(curso)				edge	(cursoCreditos)
	
	(departamento)		edge	(departamentoIdDpto)
	(departamento)		edge	(departamentoNombre)

	(docente)			edge	(docenteEmail)
	(docente)			edge	(docentePaginaWeb)
	
	(empleado)			edge	(empleadoCodigo)
	(empleado)			edge	(empleadoNombre)
	(empleado)			edge	(empleadoTelefono)		
	
	(estudiante)		edge	(estudianteCodigoEstudiante)
	(estudiante)		edge	(estudianteMail)
	(estudiante)		edge	(estudianteNombre)
	
	(inscripcion)		edge	(inscripcionNroInscripcion)
	
	(seccion)			edge	(seccionIdSeccion)
	(seccion)			edge	(seccionNombre)

	(semestre)			edge	(semestreIdSemestre)
	(semestre)			edge	(semestreFechaInicio)
	(semestre)			edge	(semestreFechaFin)
	
	(tecnico)			edge	(tecnicoNivelEstudio)
	
	(inhAdministrativo)	edge	(administrativo)
	(inhAdministrativo)	edge	(secretario)
	(inhAdministrativo)	edge	(coordinador)
	(inhAdministrativo)	edge	(tecnico)

	(inhDocente)		edge	(docente)
	(inhDocente)		edge	(aTiempoCompleto)
	(inhDocente)		edge	(aTiempoParcial)

	(inhEmpleado)		edge 	(empleado)
	(inhEmpleado)		edge	(administrativo)
	(inhEmpleado)		edge	(docente)

	(inhEstudiante)		edge	(estudiante)
	(inhEstudiante)		edge	(regular)
	(inhEstudiante)		edge	(graduado)

		
%	Atributos relaciones	
	(egresadoDe) 	edge 	(egresadoDeAnioEgreso)
	(perteneceA)	edge	(perteneceAAnioIngreso)	
	
	(para)		edge	(paraNota)
%% 	Ejes relaciones
	(dictadoEn)	edge[out=west, in=320]		node[above right, near end] {N}		(curso)
	(dictadoEn)	edge						node[above, near end] {M}		(seccion)
						
	(egresadoDe)	edge[in=south, out=north] 	node[left, xshift=-0.125cm, pos=0.8]		{N}	(graduado)
	
	(encargadoDe)	edge[out=south, in=45]	node[above left, pos=0.9] 	{1}	(departamento)
			
	(para)			edge node[right,  near end]{N}	(inscripcion)
	(para)			edge[in=north, out=west]			(lcodCurso.center)

	(perteneceA)	edge	node[right, near end]		{M}	(estudiante)
	
	(realiza)		edge	node[left,near end]		{M}	(inscripcion)
	
	(trabajaEn)		edge	node[right, near end]		{N}	(empleado)
	(trabajaEn)		edge	node[right, near end]		{1}	(departamento)
						
	(seDivideEn)	edge						node[right, near end]		{1}	(departamento)
	(seDivideEn)	edge[in=east, out=south]	node[below right, near end]		{N}	(seccion)
	
	%% PATH AUXILIARES
	(dictadoPor)	edge 	(pDictadoPorADocente.center)
;
%		
\path[rightOptional]

	(dictadoPor)				edge					node[right, near end]		{N} (curso)

	(pDictadoPorADocente.center)edge[out=east, in=330]	node[above right, pos=0.975]{1} (docente)
	(dictadoPor)				edge[out=west, in=east]	node[above, near end]		{M} (semestre)
	
	(egresadoDe)	edge[in=130, out=south] 	node[xshift=-0.35cm, pos=0.95]	{M}	(departamento)
	
	(encargadoDe)	edge						node[right, near end] 		{1}	(docente)
	
	(para)			edge	node[right, near end]	{1}	(curso)
	
	(lcodCurso.center)	edge[in=north, out=south]	node[right, yshift=-0.2cm, xshift=0.2cm, near end]	{1} (semestre)
	
	(perteneceA)	edge[in=west, out=south]	node[below, yshift=-0.1cm, pos=0.95]	{N}	(departamento)

	(realiza)		edge[in=250, out=north]	node[above,near end]	{1} (estudiante)
;
		\end{tikzpicture}	
}

\end{center}

\begin{multicols}{2}
\noindent\textbf{MER}

Departamento(\PK{idDpto}, nombre, \FK{codigoEmpleado})

Empleado(\PK{codigoEmpleado}, nombre, \FK{idDpto})

Telefonos(\FPK{codigoEmpleado}, \PK{telefono}, tipoEmpleado)

Docente(\FPK{codigoEmpleado}, email, paginaWeb, tipoDocente)

Administrativo(\FPK{codigoEmpleado}, tipoAdministrativo)

Tecnico(\FPK{codigoEmpleado}, nivelEstudio)

Coordinador(\FPK{codigoEmpleado}, email)

Estudiante(\PK{codEstudiante}, nombre, mail)

Regular(\FPK{codEstudiante})

Egresado(\FPK{codEstudiante})

Inscripcion(\FPK{codEstudiante}, \PK{nroInscripcion})

Curso(\PK{codCurso}, nombre, creditos)

Semestre(\PK{idSemestre}, fechaInicio, fechaFin)

Seccion(\PK{idSeccion}, nombre, \FK{idDpto})

egresadoDe(\FPK{codigoEstudiante}, \FPK{idDpto})

para(\FPK{codEstudiante,nroIncripcion}, \FPK{codCurso}, \FK{idSemestre}, nota)

dictadoPor(\FPK{codCurso}, \FPK{idSemestre}, \FK{codigoEmp})

dictadoEn(\FPK{curso},\FPK{idSeccion})

\paragraph{Restricciones adicionales:}
\begin{itemize}
	\item Todos los \textit{codigoEstudiante} en \textit{egresadoEn} son estudiantes graduados
	\item \textit{codEmpleado} en \textit{Departamento} es el código de un docente.
	\item \textit{codEmpleead} en \textit{dictadoPor} es el código de un docente.
	\item El docente que dicta un curso debe pertenecer a alguno de los departamentos en donde se dicta.
	\item \red{Un curso puede ser dictado en una única sección de un departamento. Osea, si hay dos secciones que dictan el mismo curso, entonces son de distinto departamentos.}
\end{itemize}

\columnbreak
\paragraph{Consultas}

\begin{enumerate}[(a)]
	\item Se pueden conseguir todas las inscpriciones realizadas por el alumno $A$ usando la relación \textit{para} y agruparlas por semestre.
	
	\item Se puede calcular a partir de la relación \textit{para}, consiguiendo todas las notas del alumno.
	
	\item Se consiguen todas las inscripciones del alumno usando la relación \textit{para}. Para cada par \textit{(semestre, curso)} devuelto, se consigue la entrada correspondiente en \textit{dictadoPor}. De acá, conseguimos el docente que dictó cada curso y, de él, el departamento en el que trabaja.
	
	Se filtran todos los cursos que corresponden al departamento $D$. 
	
	\item Idem (c) pero se devuelven todos los departamentos conseguidos (eliminando repetidos).
	
	\item Se buscan las inscripciones del alumno $A$ que corresponden al curso $C$ en la relación \textit{para}. De aquí se consigue los semestres en los que cursó. Usando la relación \textit{dictadoPor} se consigue el docente y, de ahí, el departamento correspondiente.
	
	\item Se buscan todas las inscripciones correspondientes al curso $C$ en el semestre deseado y calculamos el promedio.
	
	Si es necesario chequear que el curso pertenece a la sección $S$, entonces solo hay que ver que $S$ pertenezca al departamento que la está dictando ese semestre.
	
	\item Se consigue filtrando todas las entradas que corresponden al semestre y al profesor indicado en la relación \textit{dictadoPor}.	
\end{enumerate}

\paragraph{\red{Notas}}
\begin{itemize}
	\item \red{En una de los puntos aparece una sección salvaje que no se aclara bien lo que es, por ahora, asumo que son las áreas de cada departamento, por ejemplo, en Computación, tenemos Algoritmos, Sistemas y Teoría.}
	\item Por como hice el modelo, un curso puede ser dictado por un único departamento en cada cuatrimestre.
\end{itemize}

\end{multicols}

\subsection{Ejercicio 2.5}

\vspace*{-1.25cm}
\begin{center}
	\scalebox{0.8}{
		\begin{tikzpicture}[.style={DER}]	
	
		\end{tikzpicture}	
}
	
\end{center}

\begin{multicols}{2}
	\noindent\textbf{MER}
	
	\paragraph{Restricciones adicionales:}

	\paragraph{Consultas}
	
	\paragraph{\red{Notas}}
\end{multicols}
\end{document}