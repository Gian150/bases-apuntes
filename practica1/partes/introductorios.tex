\section{Ejercicios introductorios}

\subsection{Ejercicio 1.1}
\begin{center}
\scalebox{1}{
	\begin{tikzpicture}[DER]
	
	%% Entidades; y relaciones
	\node[entity]		(cliente)								{Cliente};
	\node[relationship]	(posee)			[right=of cliente]			{posee};
	\node[entity]		(vehiculo)		[right=of posee]		{Vehiculo};
	\node[relationship]	(cubre)			[above=of vehiculo]	{cubre};
	\node[entity] 		(poliza) 	[above=of cubre]		{Poliza};
	\node[weak relationship]	(paga)		[left=of poliza]		{paga};
	\node[weak entity]		(pago)		[left=of paga]	{Pago};
	
	%%Atributos
	\node[attributePK]	(clienteIdCliente)	[left=of cliente] {\PK{idCliente}};

	\node[attributePK]	(vehiculoIdVehiculo)	[right=of vehiculo] {\PK{idVehiculo}};	
	
	\node[attributePK]	(polizaIdPoliza)	[right=of poliza] {\PK{idPoliza}};
	
	\node[attributePK]  (pagoNroPago)		[above=of pago]	{\PK{nroPago}};
	
	\node[attribute]	(pagoFechaRecepcion)		[above left=of pago] {fechaRecepcion};
	
	\node[attribute]	(pagoPeriodo)		[left=of pago] {periodo};
	
	\node[attribute]	(pagoPeriodoInicio)		[left=of pagoPeriodo] {fechaInicio};
	
	\node[attribute]	(pagoPeriodoFin)		[below left=of pagoPeriodo, xshift=-0.5cm] {fechaFin};
	
	\node[attribute]	(pagoVencimiento)		[below left=of pago] {vencimiento};
	
	\path
		(pago) 			edge (pagoNroPago)
		(pago) 			edge (pagoFechaRecepcion)
		(pago) 			edge (pagoPeriodo)
		(pago) 			edge (pagoVencimiento)
		(pagoPeriodo) 	edge (pagoPeriodoInicio)
		(pagoPeriodo) 	edge (pagoPeriodoFin)
		
		(poliza)		edge (polizaIdPoliza)
		
		(vehiculo)		edge (vehiculoIdVehiculo)
		
		(cliente)		edge (clienteIdCliente)
		
		(paga) edge node[above, near end]{N} (pago)
		(paga) edge node[above, near end]{1} (poliza)
		
		(cubre) edge node[left, near end]{N} (poliza)
		(cubre) edge node[left, near end]{M} (vehiculo)
		
		(posee) edge node[above, near end]{1} (cliente)
		(posee) edge node[above, near end]{N} (vehiculo)
		;
	\end{tikzpicture}
	
}
\vspace*{1cm}

\begin{minipage}{0.5\textwidth}
\textbf{MER}
	
	Cliente		(\PK{idCliente})
	
	Vehiculo	(\PK{idVehiculo}, \FK{idCliente}, \FK{idPoliza})
	
	Accidente	(\PK{idAccidente}, \FK{idVehiculo})
	
	Poliza		(\PK{idPoliza})
	
	Pago		(\FPK{idPoliza}, \PK{nroPago}, fechaRecepcion, fechaInicio, fechaFin, vencimiento)
	
\end{minipage}

\end{center}

\newpage

\subsection{Ejercicio 1.2}
\begin{center}
	\scalebox{1}{
		\begin{tikzpicture}[DER]
		
		%% Entidades; y relaciones
		\node[entity] 		(cliente)							{Cliente};
		\node[relationship] (compro)		[right=of cliente]	{compró};
		\node[entity]		(producto)		[right=of compro]	{Producto};
		\node[relationship]	(provee)		[above=of producto] {provee};
		\node[entity]		(proveedor)		[above=of provee]	{Proveedor};
		
		
		\node[attributePK] (clienteDni)		[above=of cliente]  {\PK{dni}};
		\node[attributeM]  (clienteNombres)	[above left=of cliente]  {nombres};
		\node[attributeM]  (clienteApellidos)	[left=of cliente]  {apellidos};
		\node[attribute]  (clienteDireccion)	[below left=of cliente]  {dirección};
		\node[attribute]  (clienteNacimiento)	[below=of cliente]  {nacimiento};
		
		\node[attributePK] (productoCodigo)		[above right=of producto] {\PK{código}};
		\node[attribute] (productoNombre)		[right=of producto] {código};
		\node[attribute] (productoPrecio)	[below right=of producto] {precio};
		
		\node[attributePK] (proveedorCuit)		[above left=of proveedor] {\PK{cuit}};
		\node[attribute] (proveedorNombre)		[above=of proveedor] {nombre};
		\node[attribute] (proveedorDireccion)	[above right=of proveedor] {dirección};
		\path
			(cliente) edge	(clienteDni)
			(cliente) edge	(clienteNombres)
			(cliente) edge	(clienteApellidos)
			(cliente) edge	(clienteDireccion)
			(cliente) edge  (clienteNacimiento)
			
			(producto) edge	(productoCodigo)
			(producto) edge (productoNombre)
			(producto) edge	(productoPrecio)
			
			(proveedor) edge	(proveedorCuit)
			(proveedor) edge	(proveedorNombre)
			(proveedor) edge	(proveedorDireccion)
			(compro) edge node[above, near end]{N}	(cliente)
			(compro) edge node[above, near end]{M}	(producto)
			
			(provee) edge node[left, near end] {1}	(proveedor)
			(provee) edge node[left, near end] {N}	(producto)
		;
		\end{tikzpicture}
		
	}
	\vspace*{1cm}
	
	\begin{minipage}{0.5\textwidth}
		\textbf{MER}

Cliente		(\PK{dni}, direccion, nacimiento)

Nombre		(\FPK{dni}, \PK{nombre})

Apellido	(\FPK{dni}, \PK{apellido})

Producto	(\PK{codigo}, nombre, precio, \FK{cuit})

Proovedor	(\PK{cuit}, nombre, direccion)

compro		(\FPK{dni}, \FPK{codigo})

		
	\end{minipage}
	
\end{center}

\newpage
\subsection{Ejercicio 1.3}
\begin{center}
	\textbf{DER}

	\vspace{1cm}

\scalebox{0.85}{
		\begin{tikzpicture}[.style={DER}]
				
		%% Entidades; y relaciones
		\node[entity]		(paquete)								{Paquete};
		\node[relationship]	(transporta)	[left=of paquete]		{transporta};
		\node[entity]		(camionero)		[left=of transporta]	{Camionero};
		\node[relationship]	(enviadoA)		[below=of paquete]		{enviado a};
		\node[weak entity]		(direccion)		[below=of enviadoA]		{direccion};
		\node[weak relationship]	(perteneceA)	[below=of direccion]	{pertenece a};			
		\node[entity] 		(provincia) 	[below=of perteneceA]		{Provincia};
		\node[relationship]	(conducidoPor)	[left=of camionero]		{conducido por};
		\node[entity]		(camion)		[left=of conducidoPor]	{Camion};
		\node[relationship]	(es)			[below=of camion]		{es};
		\node[entity]		(modelo)		[below=of es]			{Modelo};
		\node[relationship]	(viveEn)		[left= of direccion]	{vive en};
		
		%% Atributos Camion
		\node[]				(camionI)			[below=of camion]				{}; %%Auxiliar
		\node[]				(camionII)			[below=of camionI]				{}; %%Auxiliar
		\node[]				(camionIII)			[below=of camionII]				{}; %%Auxiliar
		
		\node[attributePK]	(camionMatricula)	[above=of camion]				{\PK{matricula}};
		
		%% Atributos camionero
		\node[]				(camioneroI)			[below=of camionero]		{};
		\node[]				(camioneroII)			[below=of camioneroI]		{};
		
		\node[attributePK]	(camioneroDni)			[above=of camionero]		{\PK{dni}};
		\node[attribute]	(camioneroNombre)		[right=of camioneroDni]		{nombre};
		\node[attribute]	(camioneroApellido)		[left=of camioneroI]		{apellido};
		\node[attribute]	(camioneroTelefono)		[right=of camioneroI]		{telefono};
		\node[attribute]	(camioneroSalario)		[below left=of camioneroI]	{salario};
		
		%% Atributos Paquete
		\node[attributePK]	(paqueteCodigo)			[above=of paquete]			{\PK{codigo}};
		\node[attribute]	(paqueteDestinatario)	[above right=of paquete]	{destinatario};
		\node[attribute]	(paqueteDescripcion)	[right=of paquete]	{descripcion};

		%% Atributos provincia		
		\node[attributePK]	(provinciaCodigoProvincia)	[right=of provincia]		{\PK{codigoProv}};
		\node[attribute]	(provinciaNombre)			[below right=of provincia]		{nombre};
		
		\node[attributePK]	(direccionNro)	[above right=of direccion]		{\PK{numero}};
		\node[attributePK]	(direccionCalle)	[right=of direccion]		{\PK{calle}};
		\node[attributePK]	(direccionLocalidad)	[below right=of direccion]		{\PK{localidad}};
		%%Atributos conducido por
		\node[attributePK]	(conducidoPorFecha)	[above=of conducidoPor]		{\PK{fecha}};
		
		%%
		
		\node[attributePK]	(modeloNombre)	[left=of modelo]	{\PK{nombreModelo}};
		\node[attributePK]	(modeloMarca)	[below left=of modelo]	{\PK{marca}};
		\node[attribute]	(modeloTipo)	[below=of modelo]	{tipo};
		\node[attribute]	(modeloPotencia)[below right=of modelo]	{potencia};
		
		\path
			%% Atributos camion
			(camion)	edge  	(camionMatricula)

			%% Atributos camionero
			(camionero)	edge	(camioneroDni)
			(camionero)	edge	(camioneroNombre)
			(camionero) edge 	(camioneroApellido)	
			(camionero) edge 	(camioneroTelefono)
			(camionero) edge 	(camioneroSalario.45)
			
			%% Atributos de paquete
			(paquete)	edge	(paqueteCodigo)
			(paquete)	edge	(paqueteDescripcion)
			(paquete)	edge	(paqueteDestinatario)
			
			
			%%
			(direccion)	edge	(direccionNro)
			(direccion)	edge	(direccionCalle)
			(direccion)	edge	(direccionLocalidad)
			%% Atributos de provincia
			(provincia)	edge	(provinciaCodigoProvincia)
			(provincia)	edge	(provinciaNombre)
			
			(modelo)	edge	(modeloNombre)
			(modelo)	edge	(modeloMarca)
			(modelo)	edge	(modeloTipo)
			(modelo)	edge	(modeloPotencia)
			
			%% Relacion conducidoPor
			(conducidoPor)	edge 	node[above, near end]	{N} (camion)
			(conducidoPor)	edge	node[above, near end]	{M} (camionero)
			(conducidoPor)	edge								(conducidoPorFecha)
			
			%% Relacion transporta
			(transporta)	edge 	node[above, near end] 	{N}	(paquete)
			(transporta)	edge	node[above, near end] 	{1} (camionero)
			
			%% Relacion llega
			(enviadoA)		edge	node[left, near end]	{N}	(paquete)
			(enviadoA)		edge	node[left, near end]	{1}	(direccion)
			
			(perteneceA)	edge	node[left, near end]	{N}	(direccion)
			(perteneceA)	edge	node[left, near end]	{1}	(provincia)
			
			(es)			edge	node[left, near end]	{N}	(camion)
			(es)			edge	node[left, near end]	{1} (modelo)
			
			(viveEn)	edge[in=south, out=west]	node[right, pos=0.9]	{N}	(camionero)
			(viveEn)	edge	node[above, near end]	{1}	(direccion)	
			;	
		\end{tikzpicture}
		
}
\end{center}
\newpage
\textbf{MER}

Camion	(\PK{matricula}, \FK{nombreModelo, marca})

Modelo (\PK{nombre}, \PK{marca}, tipo, potencia)

Camionero	(\PK{dni}, nombre, apellido, telefono, , salario, \FK{numero, calle, localidad, codigoProv})
	
Paquete	(\PK{codigo}, descripcion, destinatario, \FK{numero, calle, localidad, codigoProv}, \FK{dni})
	
Direccion(\PK{numero}, \PK{calle}, \PK{localidad}, \FPK{codigoProv})

Provincia (\PK{codigoProvincia}, nombre)
	
conducidoPor(\FPK{matricula}, \FPK{dni}, \PK{fecha})


\newpage
\subsection{Ejercicio 1.4}
\subsubsection*{a)}
	\begin{center}
		\textbf{DER}
		\vspace*{1cm}
		
		\scalebox{1}{
			\begin{tikzpicture}[.style={DER}]
			%		
			%%		%% Entidades; y relaciones
			\node[entity]		(examen)										{Examen};
			\node[relationship] (calificaA)			[right=of examen]			{califica a};
			\node[entity] 		(alumno)			[right=of calificaA]		{Alumno};
			\node[entity]		(curso)				[below=of calificaA]		{Curso};
			%% Atributos examen
			\node[attributePK]	(examenIdExamen)	[above=of examen]			{\PK{idExamen}};
			\node[attribute]	(examenFecha)		[above left=of examen]		{fecha};
			\node[attribute]	(examenNota)		[left=of examen]			{nota};
			
			%% Atributos alumno
			\node[attributePK]	(alumnoIdAlumno)	[above=of alumno]			{\PK{idAlumno}};
			\node[attribute]	(alumnoNombre)		[above right=of alumno]		{nombre};
			\node[attribute]	(alumnoApellido)	[right=of alumno]			{apellido};

			%% Atributos curso
			\node[attributePK]	(cursoIdCurso)		[left=of curso]				{\PK{idCurso}};
			\node[attribute]	(cursoNombre)		[right=of curso]			{nombre};
			
			\path	
			
			%% Atributos examen
			(examen)	edge	(examenIdExamen)
			(examen)	edge	(examenFecha)
			(examen)	edge	(examenNota)
			
			%%Atributos alumno
			(alumno)	edge	(alumnoIdAlumno)
			(alumno)	edge	(alumnoNombre)
			(alumno)	edge	(alumnoApellido)

			%%Atributos curso
			(curso)		edge	(cursoIdCurso)
			(curso)		edge	(cursoNombre)

			%% Relacion califica a
			(calificaA)	edge	node[above, near end] {N}	(examen)
			(calificaA)	edge	node[above, near end] {1}	(alumno)
			(calificaA) edge	node[right, near end] {1}	(curso)
			;		
			\end{tikzpicture}
			
		}

\end{center}

\vspace*{1cm}
\begin{multicols}{2}
		
	\textbf{MER}
	
	Examen	(\PK{idExamen}, fecha, nota)
	
	Curso	(\PK{idCurso}, nombre)
	
	Alumno	(\PK{idAlumno}, nombre, apellido)
	
	califica a(\FPK{idAlumno}, \FPK{idExamen}, \FK{idCurso})
	
	\columnbreak
	
	\paragraph{Nota:} Este modelo permite que una materia tome más de un parcial a un alumno en la misma fecha. ¿Como arreglaría esto?.

\end{multicols}
\newpage
\subsubsection*{b)}
	\begin{center}
		\textbf{DER}
		\vspace*{1cm}
		
		\scalebox{1}{
			\begin{tikzpicture}[.style={DER}]	
			%% Entidades y realaciones
			\node[entity]		(curso)														{Curso};
			\node[relationship]	(evaluaA)			[right=of curso]						{califica a};
			\node[entity]		(alumno)			[right=of evaluaA]						{Alumno};

			%% Atributos curso
			\node[attributePK]	(cursoIdCurso)		[above=of curso]						{\PK{idCurso}};
			\node[attribute]	(cursoNombre)		[left=of curso]							{nombre};
%			
			%% Atributos alumno			
			\node[attributePK]	(alumnoIdAlumno)	[above=of alumno]						{\PK{idAlumno}};
			\node[attribute]	(alumnoNombre)		[above right=of alumno]					{nombre};
			\node[attribute]	(alumnoApellid)		[right=of alumno]						{apellido};
			
			%% Atributos evalua a
			\node[attributePK] 	(evaluaFecha)		[below left=of evaluaA]					{\PK{fecha}};
			\node[attribute] 	(evaluaNota)		[below  right=of evaluaA]				{nota};

			\path	
				(alumno)			edge	(alumnoIdAlumno)
				(alumno)			edge	(alumnoNombre)
				(alumno)			edge	(alumnoApellido)
				
				(curso)				edge	(cursoIdCurso)
				(curso)				edge	(cursoNombre)
				
				(evaluaA)	edge	node[above, near end] {N} (alumno)
				(evaluaA) edge	node[above, near end] {M} (curso)
				
				(evaluaA) edge	(evaluaFecha)
				(evaluaA) edge	(evaluaNota)
			;		
			
			\end{tikzpicture}
			
		}

\end{center}
	
	\vspace*{1cm}

\begin{multicols}{2}
	\textbf{MER}

Examen	(\PK{idExamen}, fecha, nota)

Curso	(\PK{idCurso}, nombre)

califica a(\FPK{idAlumno}, \FPK{idCurso}, \PK{fecha}, nota)

\columnbreak

\paragraph{Nota:} Este modelo soluciona el problema del anterior. Ahora una materia solo puede tomar un exámen por fecha a cada alumno.
\end{multicols}

\newpage
\subsection{Ejercicio 1.5}
\begin{center}
			\textbf{DER}
	\vspace*{1cm}
	
	\scalebox{1}{
		\begin{tikzpicture}[.style={DER}]	
		%% Entidades y realaciones
		\node [entity] (agregacionClienteDireccion) {
			\begin{tikzpicture}
				\node[entity]		(cliente)											{Cliente};
				\node[relationship]	(enviaA)				[right=of cliente]			{envia a};
				\node[entity]		(direccion)				[right=of enviaA]			{Direccion};
	
				%%Atributos cliente
				\node[attributePK]	(clienteNroCliente)		[above=of cliente] 			{\PK{nroCliente}};
				\node[attribute] 	(clienteSaldo)			[above left=of cliente]		{saldo};
				\node[attribute] 	(clienteLimiteCredito)	[left=of cliente]			{limiteCredito};
				\node[attribute] 	(clienteDescuento)		[below left=of cliente]		{descuento};
	
				%% Atributos direccion
				\node[attributePK]	(direccionIdDireccion)	[above=of direccion] 		{\PK{idDireccion}};
				\node[attribute] 	(direccionN)			[above right=of direccion]	{numero};
				\node[attribute] 	(direccionCalle)		[right=of direccion]			{calle};
				\node[attribute] 	(direccionComuna)		[below right=of direccion]	{comuna};
				\node[attribute] 	(direccionCiudad)		[below=of direccion]		{ciudad};
				
				\path
				%%Atributos
				(cliente)	edge	(clienteNroCliente)
				(cliente) 	edge	(clienteSaldo)
				(cliente) 	edge	(clienteLimiteCredito)
				(cliente) 	edge	(clienteDescuento)
				
				(direccion)	edge	(direccionIdDireccion)
				(direccion)	edge	(direccionN)	
				(direccion)	edge	(direccionCalle)	
				(direccion)	edge	(direccionComuna)	
				(direccion)	edge	(direccionCiudad)	
				
				%%Relaciones
				(enviaA) 	edge	node[yshift=0.2cm, pos=0.8] {N} 	(cliente)
				(enviaA) 	edge	node[yshift=0.2cm, near end] {M} 	(direccion)
				
	;
			\end{tikzpicture}
		};
	
		\node[relationship]	(esCabeceraDe)	[below=of agregacionClienteDireccion] 	{cabecera de};
		\node[entity]		(pedido)		[below=of esCabeceraDe]					{Pedido};
		\node[] 			(auxI)			[right=of pedido] 						{};  %%Auxiliar
		\node[relationship] (pide)			[right=of auxI]							{pide};
		\node[] 			(auxII)			[right=of pide] 						{};  %%Auxiliar
		\node[]				(auxIII)		[right=of auxII]						{};	 %%Auxiliar
		\node[entity]		(articulo)		[right=of auxIII]						{Articulo};
		\node[relationship]	(distribuye)	[above=of articulo]						{distribuye};
		\node[entity]		(fabrica)		[above=of distribuye]					{Fabrica};
	
	
		%%Atributos Articulo
		\node[attributePK]	(articuloNroArticulo)	[right=of articulo] 					{\PK{nroArticulo}};
		\node[attribute]	(articuloDescripcion)	[above right=of articulo] 				{descripcion};

		%%Atributos Fabrica
		\node[attributePK]	(fabricaNroFabrica)		[above=of fabrica] 						{\PK{nroFabrica}};
		\node[attribute]	(fabricaTelefono)		[above right=of fabrica] 				{telefono};
		
		%%Atributos de pedido
		\node[attributePK]	(pedidoIdPedido)		[left=of pedido] 						{\PK{idPedido}};
		\node[attribute]	(pedidoFecha)			[above left=of pedido] 					{fecha};
		
		
		%% Atributos relación fabrica distribuye articulo
		\node[attribute]	(distribuyeCantDisponible)	[left=of distribuye] 	{cantDisponible};
		
		%%Atributos pedido pide articulo
		\node[attribute]	(pideCantidad)				[below=of pide] 			{cantidad};
		
		%% Atributos PedidoFecha
		\node[attribute] (pedidoFechaDia)	[above left=of pedidoFecha, yshift=-1cm] {dia};
		
		\node[attribute] (pedidoFechaHora)	[below left=of pedidoFecha, yshift=1cm] {hora};
		\path
			%%Atributos
			(articulo)		edge	(articuloNroArticulo)
			(articulo)		edge	(articuloDescripcion)

			(pedido)		edge 	(pedidoIdPedido)
			(pedido)		edge	(pedidoFecha)
			
			(pedidoFecha)	edge	(pedidoFechaDia)
			(pedidoFecha)	edge	(pedidoFechaHora)
			
			(fabrica)		edge	(fabricaNroFabrica)
			(fabrica)		edge	(fabricaTelefono)
						
			(pide) 			edge 	(pideCantidad)
			
			(distribuye)	edge	(distribuyeCantDisponible)
			
			
			%% Relaciones
			(distribuye) 		edge	node[right, near end] {M}	(articulo)
			
			(esCabeceraDe)		edge	node[right, near end] {M}	(pedido)
			
			(pide)				edge	node[above, near end] {N} 	(pedido)
			(pide)				edge 	node[above, near end] {M} 	(articulo)
			;
			
			\path[rightOptional]
				(esCabeceraDe) 	edge 	node[right, near end] {1} 	(agregacionClienteDireccion)
				(distribuye) 	edge	node[right, near end] {N}	(fabrica)
			;
		\end{tikzpicture}
		
	}
\end{center}

\textbf{Restricciones adicionales:}
\begin{itemize}
	\item \textit{Cliente.limiteCredito} debe ser menor a 3.000.000
\end{itemize}

\begin{multicols}{2}
	\textbf{MER}
	
	Cliente(\PK{nroCliente}, saldo, limiteCredito, descuento)
	
	Direccion(\PK{idDireccion}, numero, calle, comuna, ciudad)
	
	enviaA(\FPK{nroCliente}, \FPK{idDireccion})
	
	Pedido(\PK{idPedido}, dia, hora, \FK{nroCliente, idDireccion})
	
	Articulo(\PK{nroArticulo}, descripcion)
	
	pide(\FPK{idPedido}, \FPK{nroArticulo}, cantidad)
	
	Fabrica(\PK{nroFabrica}, telefono)
	
	distribuye(\FPK{nroFabrica}, \FPK{nroArticulo}, cantDisponible)
	
	\columnbreak
	\red{\textbf{Notas}}
	\begin{itemize}
		\item Al realizar una agregación de una relación M-N, se la toma como a una entidad más. Puede participar en cualquier tipo de relación y cuantas relaciones sea necesario.
		\item Se pide saber cuantos artículos en total provee la fábrica. No es necesario almacenar este dato, se puede hacer una búsqueda en la tabla \textit{distribuye} de todos los objetos que tienen el \textit{nroFabrica} adecuado y sumar todos los valores.
		\item Fábrica tiene participación parcial en la relación distribuye. Esto quiere decir que habrá algunas qué no distribuirán ningún producto a la empresa (estas son las fábricas alternativas).
	\end{itemize}
	
\end{multicols}

\newpage
\subsection{Ejercicio 1.6}
\subsubsection*{a)}
\begin{center}
	\textbf{DER}
	\vspace*{1cm}
	
	\scalebox{1}{
		\begin{tikzpicture}[.style={DER}]	
		%% Entidades y realaciones
		\node[entity]		(programador)												{Programador};
		\node[relationship] (conoce)					[right=of programador]			{conoce};
		\node[entity]		(lenguaje)					[right=of conoce]				{Lenguaje};
		\node[entity]		(examen)					[above=of conoce]				{Examen};
		
		%%Atributos programador
		\node[attributePK]	(programadorIdProgramador)	[left=of programador]		{\PK{idProgramador}};
		\node[attribute]	(programadorNombre)			[below left=of programador]		{nombre};
		\node[attribute]	(programadorApellido)		[below=of programador]			{apellido};
		
		%%Atributos lenguaje
		\node[attributePK]	(lenguajeIdLenguaje)		[right=of lenguaje]			{\PK{idLenguaje}};
		\node[attribute]	(lenguajeNombre)			[below right=of lenguaje]		{nombre};
		
		%%Atributos examen
		\node[attribute]	(examenDificultad)			[above= of examen]				{dificultad};
		\node[attribute]	(examenCantEjercicios)		[left=of examenDificultad]		{cantEjercicios};
		\node[attributePK]	(examenIdExamen)			[left=of examen]				{\PK{idExamen}};
		\node[attribute]	(examenFechaCreacion)		[right=of examenDificultad]		{fechaCreacion};
		\node[attribute]	(examenTexto)				[right=of examen]			{texto};
		
		%%Atributos examen evalua programador
		\node[attribute]	(conoceNota)	[below=of conoce]							{nota};
		
		
		\path
			%%Atributos
			(programador)	edge 	(programadorIdProgramador)
			(programador)	edge 	(programadorNombre)
			(programador)	edge 	(programadorApellido)
			
			(lenguaje)		edge 	(lenguajeIdLenguaje)
			(lenguaje)	 	edge 	(lenguajeNombre)
			
			(examen)		edge 	(examenIdExamen)
			(examen)		edge 	(examenCantEjercicios)
			(examen)		edge 	(examenDificultad)
			(examen)		edge	(examenFechaCreacion)
			(examen)		edge	(examenTexto)
			
			(conoce)		edge	(conoceNota)
			
			%%Relaciones
			(conoce)		edge				node[above, near end]	{N}	(programador)
			(conoce)	edge	node[right, near end]{1}	(examen)	
			;
		
		\path[rightOptional]
			(conoce)	edge	node[above, near end]{1}	(lenguaje)

		;
		
	\end{tikzpicture}
	}
\end{center}

\begin{multicols}{2}

\textbf{MER}
	
	Programador(\PK{idProgramador}, nombre, apellido)
	
	Lenguaje	(\PK{idLenguaje}, nombre)
	
	Examen	(\PK{idExamen}, cantEjercicios, dificultad, fechaCreacion, texto)

	conoce(\FPK{idProgramador}, \FPK{idLenguaje}, \FK{examen}, nota)
	
\columnbreak
\red{\textbf{Notas}}
\begin{itemize}
	\item Las relaciones ternarias pueden tener atributos siempre y cuando\textbf{ no } sean identificatorios
\end{itemize}
\null\vfill
\end{multicols}

\newpage
\subsubsection*{b)}
\begin{center}
	\textbf{DER}
	\vspace*{1cm}
	
	\scalebox{1}{
		\begin{tikzpicture}[.style={DER}]
		\node[entity]		(programador)												{Programador};
		\node[relationship] (conoce)					[right=of programador]			{conoce};
		\node[entity]		(lenguaje)					[right=of conoce]				{Lenguaje};
		\node[relationship]	(evaluaA)					[above=of programador]			{evalua a};
		\node[entity]		(examen)					[above=of evaluaA]				{Examen};
		\node[relationship] (esSobre)					[above=of lenguaje]				{es sobre};
		%%Atributos programador
		\node[attributePK]	(programadorIdProgramador)	[left=of programador]		{\PK{idProgramador}};
		\node[attribute]	(programadorNombre)			[below left=of programador]		{nombre};
		\node[attribute]	(programadorApellido)		[below=of programador]			{apellido};
					
		%%Atributos lenguaje
		\node[attributePK]	(lenguajeIdLenguaje)		[right=of lenguaje]			{\PK{idLenguaje}};
		\node[attribute]	(lenguajeNombre)			[below right=of lenguaje]		{nombre};

		%%Atributos examen
		\node[attribute]	(examenDificultad)			[above= of examen]				{dificultad};
		\node[attribute]	(examenCantEjercicios)		[left=of examenDificultad]		{cantEjercicios};
		\node[attributePK]	(examenIdExamen)			[left=of examen]				{\PK{idExamen}};
		\node[attribute]	(examenFechaCreacion)		[right=of examenDificultad]		{fechaCreacion};
		\node[attribute]	(examenTexto)				[below left=of examen]			{texto};
	
		%% Atributos evaluaA
		\node[attribute]	(evaluaANota)				[right=of evaluaA]			{nota};
		\path
		%%Atributos
			(programador)	edge 	(programadorIdProgramador)
			(programador)	edge 	(programadorNombre)
			(programador)	edge 	(programadorApellido)
				
			(lenguaje)		edge 	(lenguajeIdLenguaje)
			(lenguaje)	 	edge 	(lenguajeNombre)

			(examen)		edge 	(examenIdExamen)
			(examen)		edge 	(examenCantEjercicios)
			(examen)		edge 	(examenDificultad)
			(examen)		edge	(examenFechaCreacion)
			(examen)		edge	(examenTexto)
		
			(evaluaA)		edge	(evaluaANota)
		%Relaciones
			(conoce)	edge	node[above, near end]	{N}	(programador)
			(esSobre)	edge[in=east, out=north]	node[above, pos=0.95]	{M}	(examen)
		;
					
		\path[rightOptional]
			(conoce)	edge	node[above, near end]	{M}	(lenguaje)
			(evaluaA)	edge	node[right, near end]	{N}	(programador)
			(evaluaA)	edge	node[right, near end]	{N}	(programador)
			(evaluaA)	edge	node[right, near end]	{M}	(examen)
			(esSobre)	edge	node[right, near end]	{1}	(lenguaje)
		;

%		
%		%%Relaciones
%		(conoce)		edge				node[above, near end]	{N}	(programador)
%		(conoce)	edge	node[right, near end]{1}	(examen)	
%		;
%		
%		\path[rightOptional]
%		(conoce)	edge	node[above, near end]{1}	(lenguaje)
%		
%		;
%		
		\end{tikzpicture}
	}
\end{center}

\begin{multicols}{2}
	
	\textbf{MER}
	
	Programador(\PK{idProgramador}, nombre, apellido)
	
	Lenguaje	(\PK{idLenguaje}, nombre)
	
	Examen	(\PK{idExamen}, cantEjercicios, dificultad, fechaCreacion, texto, \FK{idLenguaje})
	
	conoce(\FPK{idProgramador}, \FPK{idLenguaje})
	
	evaluaA(\FPK{idExamen}, \FK{idProgramador}, nota)
	
	\columnbreak
	\textbf{Restricciones adicionales:}
	\begin{itemize}
		\item Los examenes que evaluan a un programador deben ser sobre los lenguajes que conoce.
	\end{itemize}
\end{multicols}

\newpage
\subsection{\red{Ejercicio 1.7 - Discutir inciso e}}
\begin{multicols}{2}
\subsubsection*{a)}
	\hspace*{2cm}
	\scalebox{1}{
		\begin{tikzpicture}[.style={DER}]	
		%% Entidades y realaciones
		\node[entity]		(empleado)									{Empleado};
		\node[relationship]	(dependeDe)			[below=of empleado]		{depende de};
	
		%%Atributos empleado	
		\node[attributePK]	(empleadoLegajo)	[above=of empleado] 	 {\PK{legajo}};
		\node[attribute]	(empleadoNombre)	[above left=of empleado] {nombre};
		\node[attribute] 	(empleadoApellido) 	[left=of empleado] 		 {apellido};
		
			\node[]				(rEmpleado)			[right=of empleado]		{}; %%Auxiliar
			
		\path
			%%Atributos
			(empleado)	edge (empleadoLegajo)
			(empleado)	edge (empleadoNombre)
			(empleado)	edge (empleadoApellido)
			
			%%Relaciones
			(dependeDe) edge[in=south,out=east] node[right,xshift=0.1cm,pos=1]{supervisor} (rEmpleado.center)
			(rEmpleado.center) edge[in=55, out=north] node[above, pos=0.95] {1} (empleado)
	
		;
				
		\path[rightOptional]
		(dependeDe)	edge node[left,xshift=-0.1cm,pos=0.25]{supervisado} node[left, near end] {M} (empleado)
		;
		\end{tikzpicture}
	}

\textbf{Restricciones}
\begin{itemize}
	\item Un empleado no puede depender de si mismo.
\end{itemize}		
\subsubsection*{b)}
\hspace*{2cm}
\scalebox{1}{
	\begin{tikzpicture}[.style={DER}]	
	%% Entidades y realaciones
	\node[entity]		(empleado)												{Empleado};
	\node[inheritance]	(inheritanceEmpleado)	[below=of empleado]				{D}; 
	\node[]				(inheritanceEmpleadoN)	[left=of inheritanceEmpleado, xshift=1cm]	{tipoEmpleado};
	\node[entity]		(jefe)	 				[below=of inheritanceEmpleado]	{Jefe};

	%% Atributos	
	\node[relationship]	(dependeDe)				[right=of inheritanceEmpleado]	{depende de};
	\node[attributePK]	(empleadoLegajo)		[above=of empleado] 	 	{\PK{legajo}};
	\node[attribute]	(empleadoNombre)		[above left=of empleado] 				{nombre};
	\node[attribute] 	(empleadoApellido) 		[left=of empleado] 		{apellido};
	
	\node[attribute]	(jefeCelular)			[left=of jefe] 					{celular};

%	
	\path
		%%Atributos
		(empleado)	edge (empleadoLegajo)
		(empleado)	edge (empleadoNombre)
		(empleado)	edge (empleadoApellido)
		
		(jefe) 		edge (jefeCelular)
		
		%% Herencias
		(inheritanceEmpleado)	edge node[midway]{O} (empleado)
		(inheritanceEmpleado)	edge (jefe)
		
		%%Relaciones
		(dependeDe) edge[in=55, out=south] node[above, pos=0.95] {1} (jefe)
	;
	\path[rightOptional]
		(dependeDe) edge[out=north, in=east] node[above, near end]{N} (empleado)
	;
	\end{tikzpicture}
}

\textbf{Restricciones}
\begin{itemize}
	\item Un jefe no puede depender de si mismo.
\end{itemize}		
\end{multicols}

\newpage
\subsubsection*{c)}
\begin{center}
	\scalebox{1}{
		\begin{tikzpicture}[.style={DER}]	
		
		%%Entidades y relaciones
		\node[entity]		(empleado)												{Empleado};
		\node[inheritance]	(inheritanceEmpleado) 	[below=of empleado]				{D}; 
			\node[]				(inheritanceEmpleadoN)	[left=of inheritanceEmpleado, xshift=1cm]	{tipoEmpleado};
		\node[entity]		(jefe)	 				[below=of inheritanceEmpleado]	{Jefe};
		\node[relationship]	(dependeDe)				[right=of inheritanceEmpleado]	{depende de};
		\node[]				(izqDependeDe)			[right=of dependeDe]			{}; %Auxiliar			
		\node[relationship]	(trabajaEn) 			[right=of izqDependeDe] 		{trabaja en};
		\node[entity]		(departamento)			[below=of trabajaEn] 			{Departamento};
		\node[relationship]	(esGerenteDe)		[left=of departamento] 			{gerente de};
		
		%% Atributos empleado
		\node[attributePK]	(empleadoLegajo)	[above=of empleado] 	{\PK{legajo}};
		\node[attribute]	(empleadoNombre)	[above left=of empleado] 			{nombre};
		\node[attribute] 	(empleadoApellido) 	[left=of empleado] 	{apellido};
		
		%%Atributos jefe
		\node[attribute]	(jefeCelular)	[left=of jefe] {celular};
		
		%%Atributos departamento
		\node[attributePK]	(departamentoCodigo)	[right=of departamento]	{\PK{codigo}};
		\node[attribute]	(departamentoNombre)	[below right=of departamento] {nombre};	
		
			
		\path
		%%Atributos
		(empleado)		edge (empleadoLegajo)
		(empleado)		edge (empleadoNombre)
		(empleado)		edge (empleadoApellido)
		
		(jefe) 			edge (jefeCelular)
		
		(departamento)	edge (departamentoCodigo)
		(departamento)	edge (departamentoNombre)
		
		%%Herencias
		(inheritanceEmpleado)	edge	node[midway]{O}	(empleado)
		(inheritanceEmpleado)	edge 					(jefe)
		
		%%Relaciones
		(dependeDe)	edge[in=north, out=south]		node[above, near end]	{1}	(jefe.50)
		
		(trabajaEn)	edge[in=east, out=north]	node[above, pos=0.95]	{N}	(empleado.15)
		(trabajaEn) 		edge 				node[left, near end]	{1} (departamento)
		
		(esGerenteDe)		edge 				node[above, near end]	{1} (departamento)
		;
%		
%		
		\path[rightOptional]
		(dependeDe.north) 	edge[in=east, out=north] 	node[below, near end]		{N}	(empleado.-15)
		(esGerenteDe.west) 	edge[in=east, out=west] 	node[below left, near end] 	{1} (jefe.-15)
		;
		\end{tikzpicture}
	}
\end{center}
\textbf{Restricciones}
\begin{itemize}
	\item Un jefe no puede depender de si mismo.
	\item El gerente de un departamento debe trabajar en ese departamento.
\end{itemize}	

\newpage
\subsubsection*{d)}
\begin{center}
	\scalebox{1}{
		\begin{tikzpicture}[.style={DER}]	
		
		%%Entidades y relaciones
		\node[entity]		(empleado)												{Empleado};
		\node[inheritance]	(inheritanceEmpleado) 	[below=of empleado]				{D}; 
		\node[entity]		(jefe)	 				[below=of inheritanceEmpleado]	{Jefe};
		\node[relationship]	(dependeDe)				[right=of inheritanceEmpleado]	{depende de};
		\node[]				(izqDependeDe)			[right=of dependeDe]			{}; %Auxiliar			
		\node[relationship]	(trabajaEn) 			[right=of izqDependeDe] 		{trabaja en};
		\node[entity]		(departamento)			[below=of trabajaEn] 			{Departamento};
		\node[relationship]	(esGerenteDe)			[left=of departamento] 			{gerente de};
		
		%% Atributos empleado
		\node[attributePK]	(empleadoLegajo)		[above left=of empleado] 		{\PK{legajo}};
		\node[attribute]	(empleadoNombre)		[left=of empleado] 				{nombre};
		\node[attribute] 	(empleadoApellido) 		[below left=of empleado] 		{apellido};
		
		%%Atributos jefe
		\node[attribute]	(jefeCelular)			[left=of jefe] {celular};
		
		%%Atributos departamento
		\node[attributePK]	(departamentoCodigo)	[right=of departamento]			{\PK{codigo}};
		\node[attribute]	(departamentoNombre)	[below right=of departamento] 	{nombre};	
		
		%%Atributos empleado trabaja en
		\node[attribute]	(trabajaEnHasta)		[right=of trabajaEn]		{hasta};
		\node[attributePK]	(trabajaEnDesde)		[above=of trabajaEnHasta]		{\PK{desde}};
		
		%%EJES
		\path
		%%Atributos
		(empleado)		edge (empleadoLegajo)
		(empleado)		edge (empleadoNombre)
		(empleado)		edge (empleadoApellido)
		
		(jefe) 			edge (jefeCelular)
		
		(departamento)	edge (departamentoCodigo)
		(departamento)	edge (departamentoNombre)
		
		(trabajaEn)		edge (trabajaEnDesde)
		(trabajaEn)		edge (trabajaEnHasta)
		%%Herencias
		(inheritanceEmpleado)	edge	node[midway]{O}	(empleado)
		(inheritanceEmpleado)	edge 					(jefe)
		
		%%Relaciones
		(dependeDe)	edge[in=north, out=south]		node[above, near end]	{1}	(jefe.50)
		
		(trabajaEn)	edge[in=east, out=north]	node[above, pos=0.95]	{N}	(empleado.15)
		(trabajaEn) 		edge 				node[left, near end]	{M} (departamento)
		
		(esGerenteDe)		edge 				node[above, near end]	{1} (departamento)
		;
		%		
		%		
		\path[rightOptional]
		(dependeDe.north) 	edge[in=east, out=north] 	node[below, near end]		{N}	(empleado.-15)
		(esGerenteDe.west) 	edge[in=east, out=west] 	node[below left, near end] 	{1} (jefe.-15)
		;
		\end{tikzpicture}
	}
\end{center}
\begin{multicols}{2}
\textbf{Restricciones}

\begin{itemize}
	\item Un jefe no puede depender de si mismo.
	\item El gerente de un departamento debe trabajar en ese departamento.
	\item Un empleado no puede trabajar en dos departamentos distintos en el mismo intervalo de tiempo (o en intervalos que se superponen).
	\item Para un empleado que actualmente esté trabajando en un departamento, el atributo \textit{trabajaEn.hasta} será nulo.
\end{itemize}

\red{
\paragraph{e)} El modelo propuesto no puede garantizar que se cumpla esta condición.
Se podría agregar una Entidad puesto y crear una ternaria con Departamento y Empleado, pero perderíamos el historial de trabajo. En este caso, la relación \textit{es gerente} desaparecería y se podría conseguir esa información consiguiendo todos los empleados que tienen una entrada con el puesto de gerencia en la nueva relación.
}
\paragraph{\red{Notas}} La relación \textit{trabajaEn} tiene un atributo \textit{hasta} que será null en varios casos. ¿Es necesario arreglar esto? Se podría tener una relación especial sin atributos para los trabajadores actuales y otra para el historial. Pero hay que ver como se guardaría la fecha en la que empezó a trabajar y, cuando alguien termina de trabajar, habría que actualizar dos tablas.
\end{multicols}
